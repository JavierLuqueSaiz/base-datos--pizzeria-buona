\documentclass[a4paper]{article} 
\usepackage{amssymb}
\setlength{\parindent}{0pt}
\usepackage[utf8]{inputenc}
\usepackage[english,spanish]{babel}
\usepackage{graphicx}
\graphicspath{ {images/} }
\usepackage{ragged2e}
\usepackage[colorlinks=true,allcolors=black]{hyperref}
\usepackage{float}
\usepackage{array}
\usepackage{longtable}
\usepackage[a4paper,top=2.5cm,bottom=2.5cm,left=2.5cm,right=2.5cm,marginparwidth=1.75cm]{geometry}
\usepackage{soul}
\usepackage{eurosym}
\usepackage{hyperref}

\title{\Huge PIZZERÍA BUONA}
\author{\LARGE GESTIÓN DE DATOS: Ciencia de Datos. Grupo 19}
\date{Curso 2022/2023}

%\pagenumbering{Roman}
\begin{document}
\begin{figure}
    \centering
    \includegraphics[width=0.3\textwidth]{UPV-Logo.jpg}
    \label{fig:logo_upv}
\end{figure}
\vspace{4mm}

\maketitle

\vspace{4mm}

\begin{figure}[H]
    \centering
    \includegraphics[width=0.6\textwidth]{Pizza_buona.jpg}
    \label{fig:logo_piz}
\end{figure}

\vfill

\begin{center}
\Large Daniel Garijo, Javier Luque, Pablo Parrilla
\end{center} 

\newpage

\renewcommand\contentsname{\LARGE Índice}

\tableofcontents \vspace{3mm}

\newpage

\section{\Huge{Primera Parte. Cassandra}}
\addtocontents{toc}{\vspace{1cm}}
\subsection{\huge{Primera Etapa. Cassandra, diagrama de clases}}

\vspace{1.5mm}

\vspace{1.5mm} Como bien tratamos en la asignatura de Bases de Datos (BDA), nuestro caso busca relacionar una base de datos creada a partir de las acciones y objetos que mueve una pizzería (Pizzería Buona). Así pues, las clases que identificamos para realizar el diagrama de clases UML fueron: Empleado, Suministro, Vehículo, Cocinero, Dependiente, Repartidor, Transporte, Entrega, Pedido, Cliente, Contenido, Pizza, Membresía, Tiene, Bebida, Ingrediente y Proveedor.  

\vspace{1.5mm} De esta forma, creamos el diagrama de clases UML mediante la herramienta Lucid Chart para mostrar las clases, sus atributos y sus relaciones, incluyendo la herencia, la asociación y la composición para representar la estructura de la pizzería. 

\subsubsection{\Large{Diagrama de clases UML}}

\begin{figure}[h]
    \centering
    \includegraphics[width=1.\textwidth]{Diagrama_Negro.png}
    \label{fig:Diagrama1}
\end{figure}

\newpage

\subsubsection{\Large{Diagrama de clases orientado a la agregación }}

\begin{figure}[h]
    \centering
    \includegraphics[width=1.\textwidth]{Diseño_final.png}
    \label{fig:Diagrama2}
\end{figure}

\href{https://lucid.app/lucidchart/153030a2-1faa-4c62-8681-491e33fba2da/edit?invitationId=inv_ad99a1d9-e574-47e7-9a09-5404c31e4277&page=Q40Za8zeHSHb#}{\textbf{Link a LucidChart}}

\subsubsection{\Large{Descripción del caso con las modificaciones realizadas}}

\vspace{1.5mm} Como definición, sabemos que la agregación es una relación entre dos clases en la que una clase (clase contenedora) incluye a otra (clase contenida) como parte de su estructura. En la agregación, la clase contenida puede existir independientemente de la clase contenedora y puede estar contenida en otras clases también. De esta forma, podemos ver que la agregación se representa en eñ diagrama de clases con una línea con un rombo vacío en el extremo de la clase contenedora que apunta a la clase contenida, de manera que el rombo vacío representa la relación de "parte-de" indicando que la clase contenida es un componente de la clase contenedora.

\vspace{1.5mm} A partir del diagrama de clases UML del trabajo de BDA, lo hemos modificado para obtener un diagrama de clases orientado a la agregación con la asociación entre las clases Cliente y Pedido. Así pues, se agrega la clase Membresía a la clase Cliente; y las clases Pizza y Bebida a la clase Pedido, cuando realmente se produce la agregación con las respectivas clases asociación Contenido y Incluye\_beb. Por otra parte, también se agrega la clase Ingrediente a la clase Pizza.

\vspace{1.5mm} Uniendo lo mencionado previamente, denotamos que una clase contenedora es la clase Pedido, siendo sus clases contenidas las clases Pizza (que a su vez es clase contenedora de la clase Ingrediente) y Bebida, y que hay otra clase contenedora que sería Cliente, teniendo la clase contenida Membresía.

\newpage

\subsection{\huge{Segunda Etapa. Diseño de la base de datos Cassandra}}

\vspace{1.5mm} En esta segunda etapa, hemos realizado el workflow a partir del diagrama de clases orientado a la agregación anterior y, teniendo en cuenta este, hemos realizado el diagrama Chebotko sobre las tablas de la pizzeria obtenidas.

\subsubsection{\Large{Workflow}} \label{Workflow}

\vspace{1.5mm}1. Mostrar los pedidos ordenados por fecha y hora descendentemente.

\vspace{1.5mm}

\hspace{4mm}1\_1. Mostrar sus pizzas ordenadas por la cantidad de cada una descendentemente.

\vspace{1.5mm}2. Mostrar los clientes ordenados por apellidos y nombre ascendentemente.

\vspace{1.5mm}

\hspace{4mm}2\_1. Mostrar sus membresías ordenadas por período mensual descendentemente.

\vspace{1.5mm}

\hspace{4mm}2\_2. Mostrar sus pedidos ordenados por valoración descendentemente.

\vspace{1.5mm}3. Mostrar los empleados que lleven más de 5 años contratados.

\vspace{1.5mm}4. Mostrar los vehículos ordenados por matrícula ascendentemente.

\vspace{1.5mm}

\hspace{4mm}4\_1. Mostrar sus repartidores (los que han usado el vehículo) ordenados por apellidos y nombre ascendentemente.

\vspace{1.5mm}5. Mostrar los tipos de pizzas ordenadas por su nombre ascendentemente.

\vspace{1.5mm}

\hspace{4mm}5\_1. Mostrar sus ingredientes ordenados por su nombre y número ascendentemente.

\vspace{1.5mm}

\hspace{12mm}5\_1\_1. Mostrar sus suministros ordenados por día y hora descendentemente.

\newpage

\subsubsection{\Large{Transformación de clases}}

\begin{figure}[H]
    \centering    \includegraphics[width=1\textwidth]{diag2.png}
    \label{fig:Diagrama3_1}
\end{figure}

\begin{figure}[H]
    \centering    \includegraphics[width=0.8\textwidth]{diag1.png}
    \label{fig:Diagrama3_2}
\end{figure}

\vspace{2mm}

\newpage

\subsubsection{\Large{Diagrama Chebotko: Queries}}

\begin{itemize}

    \item \textbf{\large{Tarea Q1}}

    \begin{itemize}

        \item RT1: Clase Pedido.

        \item RT4: Atributos ordenación fecha, hora y código.

        \item RT5: Atributos estáticos clase Pedido.

    \end{itemize}

    \begin{figure}[H]
        \centering
        \includegraphics[scale=0.4]{GDA_Q1.png}
        \label{fig:Q1}
    \end{figure}

    \item \textbf{\large{Tarea Q1\_1}}

    \begin{itemize}
    
        \item RT1: clase asociación Contenido.

        \item RT2: búsqueda igualdad cod\_ped y fecha\_ped.

        \item RT4: atributo ordenación cantidad.

        \item RT5: atributos clave num\_pizza.

    \end{itemize}

    \begin{figure}[H]
        \centering
        \includegraphics[scale=0.4]{GDA_Q1_1.png}
        \label{fig:Q1_1}
    \end{figure}

    \item \textbf{\large{Tarea Q2}}

    \begin{itemize}

        \item RT1: clase Cliente.

        \item RT4: atributos ordenación apellidos y nombre.

        \item RT5: atributos clave dni.

    \end{itemize}

    \begin{figure}[H]
        \centering
        \includegraphics[scale=0.4]{GDA_Q2.png}
        \label{fig:Q2}
    \end{figure}

\newpage

    \item \textbf{\large{Tarea Q2\_1}}

    \begin{itemize}

        \item RT1: clase asociación Tiene.

        \item RT2: búsqueda igualdad dni.

        \item RT4: atributos ordenación año y mes.

        \item RT5: atributos clave nombre\_mem.

    \end{itemize}

    \begin{figure}[H]
        \centering
        \includegraphics[scale=0.4]{GDA_Q2_1.png}
        \label{fig:Q2_1}
    \end{figure}

    \item \textbf{\large{Tarea Q2\_2}}

    \begin{itemize}

        \item RT1: asociación Realiza y clase Pedido.

        \item RT2: búsqueda de igualdad dni.

        \item RT4: atributo de ordenación valoración.

        \item RT5: atributos claves codigo y fecha.

    \end{itemize}

    \begin{figure}[H]
        \centering
        \includegraphics[scale=0.4]{GDA_Q2_2.png}
        \label{fig:Q2_2}
    \end{figure}

    \item \textbf{\large{Tarea Q3}}

    \begin{itemize}

        \item RT1: clase Empleado.

        \item RT3: búsqueda por rango años\_contratado $>$ 5.

        \item RT5: atributos clave dni.

    \end{itemize}

    \begin{figure}[H]
        \centering
        \includegraphics[scale=0.4]{GDA_Q3.png}
        \label{fig:Q3}
    \end{figure}

\newpage

    \item \textbf{\large{Tarea Q4}}

    \begin{itemize}

        \item RT1: clase Vehículo.

        \item RT4: atributo de ordenación matricula.

        \item RT5: atributos clave matricula.

    \end{itemize}

    \begin{figure}[H]
        \centering
        \includegraphics[scale=0.4]{GDA_Q4.png}
        \label{fig:Q4}
    \end{figure}

    \item \textbf{\large{Tarea Q4\_1}}

    \begin{itemize}

        \item RT1: clase asociación Conduce.

        \item RT2: búsqueda de igualdad matricula.

        \item RT4: atributo de ordenación apellidos y nombre.

        \item RT5: atributos clave dni.

    \end{itemize}

    \begin{figure}[H]
        \centering
        \includegraphics[scale=0.4]{GDA_Q4_1.png}
        \label{fig:Q4_1}
    \end{figure}

    \item \textbf{\large{Tarea Q5}}

    \begin{itemize}

        \item RT1: clase Pizza.

        \item RT4: atributo de ordenación nombre.

    \end{itemize}

    \begin{figure}[H]
        \centering
        \includegraphics[scale=0.4]{GDA_Q5.png}
        \label{fig:Q5}
    \end{figure}

\newpage

    \item \textbf{\large{Tarea Q5\_1}}

    \begin{itemize}

        \item RT1: agregación Lleva y clase Ingrediente.

        \item RT2: búsqueda de igualdad nom\_pizza.

        \item RT4: atributos de ordenación nombre y numero.

        \item RT5: atributos de clave numero.

    \end{itemize}

    \begin{figure}[H]
        \centering
        \includegraphics[scale=0.4]{GDA_Q5_1.png}
        \label{fig:Q5_1}
    \end{figure}

    \item \textbf{\large{Tarea Q5\_1\_1}}

    \begin{itemize}

        \item RT1: asociación clase Ingrediente y clase Suministro.

        \item RT2: búsqueda de igualdad numero y dia.

        \item RT4: atributo de ordenación dia y hora.

        \item RT5: atributos de clave dia y numero.

    \end{itemize}

    \begin{figure}[H]
        \centering
        \includegraphics[scale=0.4]{GDA_Q5_1_1.png}
        \label{fig:Q5_1_1}
    \end{figure}

\end{itemize}

\subsubsection{\Large{Diagrama Chebotko: Completo}}

\vspace{2mm}

\begin{figure}[H]
    \centering
    \includegraphics[scale=0.2]{Diagrama_Chebotko.png}
    \label{fig:Chebotko}
\end{figure}

\newpage

\subsection{\huge{Tercera Etapa. Creación y carga de la base de datos Cassandra}}

En esta tercera etapa, se ha realizado la creación y carga de información en la base de datos Cassandra. Para ello, se han creado las tablas obtenidas en la etapa anterior y se han recabado los datos desde Oracle mediante consultas SQL. El resultado de estas consultas se ha exportado a ficheros CSV (codificación UTF-8) para posteriormente cargarlos a su correspondiente tabla Cassandra.

\subsubsection{\Large{Instrucciones de creación de las tablas Cassandra}}

Para empezar a crear las tablas, primero es necesario crear un \emph{keyspace}.

\begin{figure}[H]
    \centering
    \includegraphics[scale=0.8]{keyspace.png}
    \label{fig:keyspace}
\end{figure}

En la creación de tablas, se han utilizado dos tipos definidos: \emph{direccion\_t} y \emph{periodo\_t}.

\begin{figure}[H]
    \centering
    \begin{minipage}[c]{0.4\textwidth}
        \centering
        \includegraphics[scale=1]{tipo_direccion.png}
    \end{minipage}
    \begin{minipage}[c]{0.4\textwidth}
        \centering
        \includegraphics[scale=1]{tipo_periodo.png}
    \end{minipage}
    \label{fig:tipos}
\end{figure}

\begin{itemize}

    \item \textbf{\large{Tabla Empleado}}
    
    \begin{figure}[H]
        \centering
        \includegraphics[scale=1]{tabla_empleado.png}
        \label{fig:tabla_empleado}
    \end{figure}
    
    \item \textbf{\large{Tabla Vehículo}}

    \begin{figure}[H]
        \centering
        \includegraphics[scale=1]{tabla_vehiculo.png}
        \label{fig:tabla_vehiculo}
    \end{figure}

\newpage

    \item \textbf{\large{Tabla Suministro}}

    \begin{figure}[H]
        \centering
        \includegraphics[scale=1]{tabla_suministro.png}
        \label{fig:tabla_suministro}
    \end{figure}

    \item \textbf{\large{Tabla Ingrediente}}

    \begin{figure}[H]
        \centering
        \includegraphics[scale=1]{tabla_ingrediente.png}
        \label{fig:tabla_ingrediente}
    \end{figure}

    \item \textbf{\large{Tabla Pizza}}

    \begin{figure}[H]
        \centering
        \includegraphics[scale=1]{tabla_pizza.png}
        \label{fig:tabla_pizza}
    \end{figure}

    \item \textbf{\large{Tabla Pedido}}

    \begin{figure}[H]
        \centering
        \includegraphics[scale=1]{tabla_pedido.png}
        \label{fig:tabla_pedido}
    \end{figure}

\newpage

    \item \textbf{\large{Tabla Membresía}}

    \begin{figure}[H]
        \centering
        \includegraphics[scale=1]{tabla_membresia.png}
        \label{fig:tabla_membresia}
    \end{figure}

    \item \textbf{\large{Tabla Cliente}}

    \begin{figure}[H]
        \centering
        \includegraphics[scale=1]{tabla_cliente.png}
        \label{fig:tabla_cliente}
    \end{figure}

    \item \textbf{\large{Tabla Pedido por fecha y hora (Q1)}}

    \begin{figure}[H]
        \centering
        \includegraphics[scale=1]{tabla_Q1.png}
        \label{fig:tabla_Q1}
    \end{figure}

    \item \textbf{\large{Tabla Pizza por cantidad (Q1\_1)}}

    \begin{figure}[H]
        \centering
        \includegraphics[scale=1]{tabla_Q1_1.png}
        \label{fig:tabla_Q1_1}
    \end{figure}

\newpage

    \item \textbf{\large{Tabla Cliente por apellidos y nombre (Q2)}}

    \begin{figure}[H]
        \centering
        \includegraphics[scale=1]{tabla_Q2.png}
        \label{fig:tabla_Q2}
    \end{figure}

    \item \textbf{\large{Tabla Membresía por año y mes (Q2\_1)}}

    \begin{figure}[H]
        \centering
        \includegraphics[scale=1]{tabla_Q2_1.png}
        \label{fig:tabla_Q2_1}
    \end{figure}

    \item \textbf{\large{Tabla Pedido por valoración (Q2\_2)}}

    \begin{figure}[H]
        \centering
        \includegraphics[scale=1]{tabla_Q2_2.png}
        \label{fig:tabla_Q2_2}
    \end{figure}

    \item \textbf{\large{Tabla Empleado más de cinco años contratado (Q3)}}

    \begin{figure}[H]
        \centering
        \includegraphics[scale=1]{tabla_Q3.png}
        \label{fig:tabla_Q3}
    \end{figure}

    \item \textbf{\large{Tabla Vehículo por matrícula (Q4)}}

    \begin{figure}[H]
        \centering
        \includegraphics[scale=1]{tabla_Q4.png}
        \label{fig:tabla_Q4}
    \end{figure}

    \item \textbf{\large{Tabla Repartidor por apellidos y nombre (Q4\_1)}}

    \begin{figure}[H]
        \centering
        \includegraphics[scale=1]{tabla_Q4_1.png}
        \label{fig:tabla_Q4_1}
    \end{figure}

    \item \textbf{\large{Tabla Pizza por nombre (Q5)}}

    \begin{figure}[H]
        \centering
        \includegraphics[scale=1]{tabla_Q5.png}
        \label{fig:tabla_Q5}
    \end{figure}

    \item \textbf{\large{Tabla Ingrediente por nombre y numero (Q5\_1)}}

    \begin{figure}[H]
        \centering
        \includegraphics[scale=1]{tabla_Q5_1.png}
        \label{fig:tabla_Q5_1}
    \end{figure}

\newpage

    \item \textbf{\large{Tabla Suministro por día y hora (Q5\_1\_1)}}

    \begin{figure}[H]
        \centering
        \includegraphics[scale=1]{tabla_Q5_1_1.png}
        \label{fig:tabla_Q5_1_1}
    \end{figure}
        
\end{itemize}

\newpage

\subsubsection{\Large{Consultas SQL para la carga de las tablas Cassandra}}

Algunas aclaraciones sobre las consultas:

\begin{enumerate}
    \item En los booleanos se ha modificado la salida de {1,0} a {`true', `false'}.
    \item Para los empleados, se ha añadido la columna tipo indicando si el empleado es repartidor o dependiente (no hay cocineros).
    \item En el diagrama original, Ingrediente es una clase débil de Proveedor, y la clase Proveedor ha sido podada. Por tanto, el código del ingrediente, que por sí mismo puede repetirse, pasa a ser la concatenación de dicho código con el nif del proveedor del ingrediente; asegurando así unicidad.
    \item Para las horas, se han añadido los segundos con la concatenación \emph{hora $\mid \mid$ `:00'}, para que el formato sea el correcto para el tipo TIME de Cassandra.
    \item Para el tipo definido \emph{direccion\_t}, existen posibles valores nulos en los atributos \textbf{esc\_dir\_cli} y \textbf{piso\_dir\_cli}, y Cassandra no admite valores nulos en atributos de tipos definidos por el usuario. Por lo tanto, para los posibles valores de estos atributos que sean nulos, se ha cambiado el valor por la cadena de texto `null', y que de esta forma, funcione a modo de valor nulo a pesar de que no lo sea estrictamente.
    
    Además, se han utilizado concatenaciones con el objetivo de dar el formato correcto para que Cassandra pueda leer el tipo definido \emph{direccion\_t}.
    \item Originalmente, los teléfonos de los clientes estaban en una tabla aparte, ya que un cliente podía tener más de 1 teléfono. Por lo tanto, se han recogido todos los teléfonos de cada cliente dándole el formato adecuado del tipo SET.
    \item Al igual que con el tipo definido \emph{direccion\_t}, se han utilizado concatenaciones para darle el formato adecuado para el tipo \emph{periodo\_t}.
    
    Además, en el caso de la consulta de Cliente, al tener agregada la clase Membresía y existir clientes que no tienen membresía, el valor de este tipo definido podría ser nulo. Por lo tanto, se ha considerado esta posibilidad en la consulta.
    \item En relación con el apartado anterior, como el nombre de la membresía puede ser nulo y en la tabla agregada Cliente se trata de un clave de clúster; se sustituye el posible valor por el valor `Ninguna'.
    
\end{enumerate}

\begin{itemize}

    \item \textbf{\large{Consulta Empleado}}
    
    \begin{figure}[H]
        \centering
        \includegraphics[scale=1]{consulta_empleado.png}
        \label{fig:consulta_empleado}
    \end{figure}

\newpage
    
    \item \textbf{\large{Consulta Vehículo}}

    \begin{figure}[H]
        \centering
        \includegraphics[scale=1]{consulta_vehiculo.png}
        \label{fig:consulta_vehiculo}
    \end{figure}

    \item \textbf{\large{Consulta Suministro}}

    \begin{figure}[H]
        \centering
        \includegraphics[scale=1]{consulta_suministro.png}
        \label{fig:consulta_suministro}
    \end{figure}

    \item \textbf{\large{Consulta Ingrediente}}

    \begin{figure}[H]
        \centering
        \includegraphics[scale=1]{consulta_ingrediente.png}
        \label{fig:consulta_ingrediente}
    \end{figure}

    \item \textbf{\large{Consulta Pizza}}

    \begin{figure}[H]
        \centering
        \includegraphics[scale=1]{consulta_pizza.png}
        \label{fig:consulta_pizza}
    \end{figure}

    \item \textbf{\large{Consulta Pedido}}

    \begin{figure}[H]
        \centering
        \includegraphics[scale=1]{consulta_pedido.png}
        \label{fig:consulta_pedido}
    \end{figure}

\newpage

    \item \textbf{\large{Consulta Membresía}}

    \begin{figure}[H]
        \centering
        \includegraphics[scale=1]{consulta_membresia.png}
        \label{fig:consulta_membresia}
    \end{figure}

    \item \textbf{\large{Consulta Cliente}}

    \begin{figure}[H]
        \centering
        \includegraphics[scale=1]{consulta_cliente.png}
        \label{fig:consulta_cliente}
    \end{figure}

\newpage

    \item \textbf{\large{Consulta Pedido por fecha y hora (Q1)}}

    \begin{figure}[H]
        \centering
        \includegraphics[scale=1]{consulta_Q1.png}
        \label{fig:consulta_Q1}
    \end{figure}

    \item \textbf{\large{Consulta Pizza por cantidad (Q1\_1)}}

    \begin{figure}[H]
        \centering
        \includegraphics[scale=1]{consulta_Q1_1.png}
        \label{fig:consulta_Q1_1}
    \end{figure}

    \item \textbf{\large{Consulta Cliente por apellidos y nombre (Q2)}}

    \begin{figure}[H]
        \centering
        \includegraphics[scale=1]{consulta_Q2.png}
        \label{fig:consulta_Q2}
    \end{figure}

\newpage

    \item \textbf{\large{Consulta Membresía por año y mes (Q2\_1)}}

    \begin{figure}[H]
        \centering
        \includegraphics[scale=1]{consulta_Q2_1.png}
        \label{fig:consulta_Q2_1}
    \end{figure}

    \item \textbf{\large{Consulta Pedido por valoración (Q2\_2)}}

    \begin{figure}[H]
        \centering
        \includegraphics[scale=1]{consulta_Q2_2.png}
        \label{fig:consulta_Q2_2}
    \end{figure}

    \item \textbf{\large{Consulta Empleado más de cinco años contratado (Q3)}}

    \begin{figure}[H]
        \centering
        \includegraphics[scale=1]{consulta_Q3.png}
        \label{fig:consulta_Q3}
    \end{figure}

\newpage

    \item \textbf{\large{Consulta Vehículo por matrícula (Q4)}}

    \begin{figure}[H]
        \centering
        \includegraphics[scale=1]{consulta_Q4.png}
        \label{fig:consulta_Q4}
    \end{figure}

    \item \textbf{\large{Consulta Repartidor por apellidos y nombre (Q4\_1)}}

    \begin{figure}[H]
        \centering
        \includegraphics[scale=1]{consulta_Q4_1.png}
        \label{fig:consulta_Q4_1}
    \end{figure}

    \item \textbf{\large{Consulta Pizza por nombre (Q5)}}

    \begin{figure}[H]
        \centering
        \includegraphics[scale=1]{consulta_Q5.png}
        \label{fig:consulta_Q5}
    \end{figure}

    \item \textbf{\large{Consulta Ingrediente por nombre y numero (Q5\_1)}}

    \begin{figure}[H]
        \centering
        \includegraphics[scale=1]{consulta_Q5_1.png}
        \label{fig:consulta_Q5_1}
    \end{figure}

\newpage

    \item \textbf{\large{Consulta Suministro por día y hora (Q5\_1\_1)}}

    \begin{figure}[H]
        \centering
        \includegraphics[scale=1]{consulta_Q5_1_1.png}
        \label{fig:consulta_Q5_1_1}
    \end{figure}
        
\end{itemize}

\subsubsection{\Large{Carga de las tablas Cassandra}}

Para realizar la carga de datos, una vez nos encontramos en el MobaXterm habiendo ejecutado el cqlsh desde la carpeta en la que se han guardado todos los ficheros CSV, todo el código a introducir por consola sigue el mismo formato:

\begin{figure}[H]
    \includegraphics[scale=0.9]{carga_formato.png}
    \label{fig:carga_formato}
\end{figure}

\begin{itemize}

    \item \textbf{\large{Tabla Empleado}}
    
    \begin{figure}[H]
        \includegraphics[scale=1]{carga_empleado.png}
        \label{fig:carga_empleado}
    \end{figure}
    
    \item \textbf{\large{Tabla Vehículo}}

    \begin{figure}[H]
        \includegraphics[scale=1]{carga_vehiculo.png}
        \label{fig:carga_vehiculo}
    \end{figure}

    \item \textbf{\large{Tabla Suministro}}

    \begin{figure}[H]
        \includegraphics[scale=1]{carga_suministro.png}
        \label{fig:carga_suministro}
    \end{figure}

    \item \textbf{\large{Tabla Ingrediente}}

    \begin{figure}[H]
        \includegraphics[scale=1]{carga_ingrediente.png}
        \label{fig:carga_ingrediente}
    \end{figure}

    \item \textbf{\large{Tabla Pizza}}

    \begin{figure}[H]
        \includegraphics[scale=1]{carga_pizza.png}
        \label{fig:carga_pizza}
    \end{figure}

    \item \textbf{\large{Tabla Pedido}}

    \begin{figure}[H]
        \includegraphics[scale=1]{carga_pedido.png}
        \label{fig:carga_pedido}
    \end{figure}

    \item \textbf{\large{Tabla Membresía}}

    \begin{figure}[H]
        \includegraphics[scale=1]{carga_membresia.png}
        \label{fig:carga_membresia}
    \end{figure}

    \item \textbf{\large{Tabla Cliente}}

    \begin{figure}[H]
        \includegraphics[scale=1]{carga_cliente.png}
        \label{fig:carga_cliente}
    \end{figure}

    \item \textbf{\large{Tabla Pedido por fecha y hora (Q1)}}

    \begin{figure}[H]
        \includegraphics[scale=1]{carga_Q1.png}
        \label{fig:carga_Q1}
    \end{figure}

    \item \textbf{\large{Tabla Pizza por cantidad (Q1\_1)}}

    \begin{figure}[H]
        \includegraphics[scale=1]{carga_Q1_1.png}
        \label{fig:carga_Q1_1}
    \end{figure}

    \item \textbf{\large{Tabla Cliente por apellidos y nombre (Q2)}}

    \begin{figure}[H]
        \includegraphics[scale=1]{carga_Q2.png}
        \label{fig:carga_Q2}
    \end{figure}

    \item \textbf{\large{Tabla Membresía por año y mes (Q2\_1)}}

    \begin{figure}[H]
        \includegraphics[scale=1]{carga_Q2_1.png}
        \label{fig:carga_Q2_1}
    \end{figure}

    \item \textbf{\large{Tabla Pedido por valoración (Q2\_2)}}

    \begin{figure}[H]
        \includegraphics[scale=1]{carga_Q2_2.png}
        \label{fig:carga_Q2_2}
    \end{figure}

    \item \textbf{\large{Tabla Empleado más de cinco años contratado (Q3)}}

    \begin{figure}[H]
        \includegraphics[scale=1]{carga_Q3.png}
        \label{fig:carga_Q3}
    \end{figure}

    \item \textbf{\large{Tabla Vehículo por matrícula (Q4)}}

    \begin{figure}[H]
        \includegraphics[scale=1]{carga_Q4.png}
        \label{fig:carga_Q4}
    \end{figure}

    \item \textbf{\large{Tabla Repartidor por apellidos y nombre (Q4\_1)}}

    \begin{figure}[H]
        \includegraphics[scale=1]{carga_Q4_1.png}
        \label{fig:carga_Q4_1}
    \end{figure}

    \item \textbf{\large{Tabla Pizza por nombre (Q5)}}

    \begin{figure}[H]
        \includegraphics[scale=1]{carga_Q5.png}
        \label{fig:carga_Q5}
    \end{figure}

    \item \textbf{\large{Tabla Ingrediente por nombre y numero (Q5\_1)}}

    \begin{figure}[H]
        \includegraphics[scale=1]{carga_Q5_1.png}
        \label{fig:carga_Q5_1}
    \end{figure}

    \item \textbf{\large{Tabla Suministro por día y hora (Q5\_1\_1)}}

    \begin{figure}[H]
        \includegraphics[scale=1]{carga_Q5_1_1.png}
        \label{fig:carga_Q5_1_1}
    \end{figure}
        
\end{itemize}

\newpage

\subsection{\huge{Cuarta Etapa. Resolución de consultas CQL}}

En esta cuarta etapa, se resolverán en CQL algunas consultas de los accesos planteados en el \hyperref[Workflow]{Workflow de la etapa 2}.

\begin{enumerate}
    \item Mostrar los pedidos ordenados por fecha y hora descendentemente.

    \begin{figure}[H]
        \centering
        \includegraphics[scale=0.8]{cons_1.png}
        \label{fig:cons_1}
    \end{figure}

    \item Mostrar las pizzas de un pedido (el segundo del 30 de diciembre del 2022) ordenadas por la cantidad de cada una descendentemente.

    \begin{figure}[H]
        \centering
        \includegraphics[scale=0.8]{cons_2.png}
        \label{fig:cons_2}
    \end{figure}

    \item Mostrar los clientes ordenados por apellidos y nombre ascendentemente.

    \begin{figure}[H]
        \centering
        \includegraphics[scale=0.8]{cons_3.png}
        \label{fig:cons_3}
    \end{figure}

    \newpage
    
    \item Mostrar las membresías de un cliente (dni 49226378F) ordenadas por período mensual descendentemente.

    \begin{figure}[H]
        \centering
        \includegraphics[scale=0.8]{cons_4.png}
        \label{fig:cons_4}
    \end{figure}

    \item Mostrar los pedidos de un cliente (dni 43175707S) ordenados por valoración descendentemente.

    \begin{figure}[H]
        \centering
        \includegraphics[scale=0.8]{cons_5.png}
        \label{fig:cons_5}
    \end{figure}

    \item Mostrar los empleados que lleven más de 5 años contratados.

    \begin{figure}[H]
        \centering
        \includegraphics[scale=0.8]{cons_6.png}
        \label{fig:cons_6}
    \end{figure}

    \newpage
    
    \item Mostrar los vehículos ordenados por matrícula ascendentemente.

    \begin{figure}[H]
        \centering
        \includegraphics[scale=0.8]{cons_7.png}
        \label{fig:cons_7}
    \end{figure}

    \item Mostrar los repartidores de un vehículo (matrícula 1768DTH); es decir, los que han usado el vehículo, ordenados por apellidos y nombre ascendentemente.

    \begin{figure}[H]
        \centering
        \includegraphics[scale=0.8]{cons_8.png}
        \label{fig:cons_8}
    \end{figure}

    \item Mostrar los tipos de pizzas ordenadas por su nombre ascendentemente.

    \begin{figure}[H]
        \centering
        \includegraphics[scale=0.8]{cons_9.png}
        \label{fig:cons_9}
    \end{figure}

    \newpage

    \item Mostrar los ingredientes de una pizza (numero PI110) ordenados por su nombre y número ascendentemente.
    
    \begin{figure}[H]
        \centering
        \includegraphics[scale=0.8]{cons_10.png}
        \label{fig:cons_10}
    \end{figure}

    \item Mostrar los suministros de un ingrediente (carne número IN004A92379420) ordenados por día y hora descendentemente.

    \begin{figure}[H]
        \centering
        \includegraphics[scale=0.8]{cons_11.png}
        \label{fig:cons_11}
    \end{figure}
    
\end{enumerate}

\newpage

\section{\Huge{Segunda Parte. Neo4j}}
\addtocontents{toc}{\vspace{1cm}}
\subsection{\huge{Primera Etapa. Diseño de la base de datos Neo4j}}

En esta primera etapa de la segunda parte, se va a proceder al diseño de la base de datos Neo4j. Este diseño se va a realizar a partir del diagrama de clases utilizado también para el diseño de la base de datos Cassandra. En esta ocasión no se contempla la agregación de las clases utilizada con Cassandra, pero sí que se mantiene la poda sugerida. Así, en la \hyperref[fig:podado]{figura siguiente} se puede observar dicho diagrama.

\subsubsection{\Large{Diagrama de clases}}

\begin{figure}[H]
    \centering
    \includegraphics[scale=0.25]{podado.png}
    \label{fig:podado}
\end{figure}

Así pues, obtenemos el diagrama lógico de la base de datos Neo4j, con los atributos de las clases y a modo de grafo \footnote{Se han cambiado el tipo de dato de algunos atributos de acuerdo a cambios realizados también en el diseño de la base de datos Cassandra} \footnote{Se han cambiado los nombres de las clases de forma verbal al infinitivo para poder usar los verbos conjugados en los nombres de los arcos}.

\newpage

\subsubsection{\Large{Diagrama Lógico Neo4j}}

\begin{figure}[H]
    \centering
    \includegraphics[scale=0.3]{dlog_neo.png}
    \label{fig:dlog_neo}
\end{figure}

\subsubsection{\Large{Diagrama Lógico Neo4j (Grafo)}}

\begin{figure}[H]
    \centering
    \includegraphics[scale=0.3]{dgraf_neo.png}
    \label{fig:dgraf_neo}
\end{figure}

\newpage

\subsection{\huge{Segunda Etapa. Creación y carga de la base de datos Neo4j}}

En esta segunda etapa se va a proceder a cargar la base de datos en Neo4j. Para ello, se han realizado consultas en Oracle para obtener los archivos CSV que posteriormente se cargarán. En estas consultas no se ha utilizado en ningún momento SELECT * y siempre se seleccionan los atributos uno a uno. Esto se ha hecho, por una parte, para cambiar los nombres de los atributos para que tengan los mismos que las propiedades de los nodos que se implementarán en Neo4j; y por otra parte, para que sea más comprensible visualmente cuales son los atributos que se exportan en cada CSV.

\vspace{1.5mm}Así, se mostrará la consulta utilizada e inmediatamente seguido el código Cypher. Por último, cada vez que se complete la carga de un grupo de nodos y/o arcos que van conjuntos en el mismo código, se mostrará el diagrama lógico en forma de grafo de cómo queda la base de datos tras cada carga.

\subsubsection{\Large{Nodo Empleado (Dependiente y Repartidor)}}

\begin{itemize}

    \item \textbf{\large{Consulta Dependiente}}
    
    \begin{figure}[H]
        \centering
        \includegraphics[scale=1]{consultan_dependiente.png}
        \label{fig:consultan_dependiente}
    \end{figure}

    \item \textbf{\large{Carga Dependiente}}
    
    \begin{figure}[H]
        \centering
        \includegraphics[scale=1]{cargan_dependiente.png}
        \label{fig:cargan_dependiente}
    \end{figure}

    \item \textbf{\large{Consulta Idiomas}}
    
    \begin{figure}[H]
        \centering
        \includegraphics[scale=1]{consultan_idiomas.png}
        \label{fig:consultan_idiomas}
    \end{figure}
    
    \item \textbf{\large{Carga Idiomas}}
    
    \begin{figure}[H]
        \centering
        \includegraphics[scale=1]{cargan_idiomas.png}
        \label{fig:cargan_idiomas}
    \end{figure}

    \item \textbf{\large{Consulta Repartidor}}

    \begin{figure}[H]
        \centering
        \includegraphics[scale=1]{consultan_repartidor.png}
        \label{fig:consultan_repartidor}
    \end{figure}

    \item \textbf{\large{Carga Repartidor}}

    \begin{figure}[H]
        \centering
        \includegraphics[scale=1]{cargan_repartidor.png}
        \label{fig:cargan_repartidor}
    \end{figure}

    \item \textbf{\large{Consulta Permisos}}

    \begin{figure}[H]
        \centering
        \includegraphics[scale=1]{consultan_permisos.png}
        \label{fig:consultan_permisos}
    \end{figure}

    \item \textbf{\large{Carga Permisos}}

    \begin{figure}[H]
        \centering
        \includegraphics[scale=1]{cargan_permisos.png}
        \label{fig:cargan_permisos}
    \end{figure}

    \item \textbf{\large{Grafo}}

    \begin{figure}[H]
        \centering
        \includegraphics[scale=0.3]{grafo_empleado.png}
        \label{fig:grafo_empleado}
    \end{figure}

\end{itemize}

\subsubsection{\Large{Nodo Ingrediente}}

\begin{itemize}

    \item \textbf{\large{Consulta Ingrediente}}

    \begin{figure}[H]
        \centering
        \includegraphics[scale=1]{consultan_ingrediente.png}
        \label{fig:consultan_ingrediente}
    \end{figure}

    \item \textbf{\large{Carga Ingrediente}}

    \begin{figure}[H]
        \centering
        \includegraphics[scale=1]{cargan_ingrediente.png}
        \label{fig:cargan_ingrediente}
    \end{figure}

    \item \textbf{\large{Grafo}}

    \begin{figure}[H]
        \centering
        \includegraphics[scale=0.3]{grafo_ingrediente.png}
        \label{fig:grafo_ingrediente}
    \end{figure}

\end{itemize}

\subsubsection{\Large{Nodo Suministro y Arcos Suministra y Recibe}}

\begin{itemize}

    \item \textbf{\large{Consulta Suministro / Suministra / Recibe}}

    \begin{figure}[H]
        \centering
        \includegraphics[scale=1]{consultan_suministro.png}
        \label{fig:consultan_suministro}
    \end{figure}

    \item \textbf{\large{Carga Suministro / Suministra / Recibe}}

    \begin{figure}[H]
        \centering
        \includegraphics[scale=1]{cargan_suministro.png}
        \label{fig:cargan_suministro}
    \end{figure}

    \item \textbf{\large{Grafo}}

    \begin{figure}[H]
        \centering
        \includegraphics[scale=0.3]{grafo_suministro.png}
        \label{fig:grafo_suministro}
    \end{figure}

\end{itemize}

\subsubsection{\Large{Nodo Pizza}}

\begin{itemize}

    \item \textbf{\large{Consulta Pizza}}

    \begin{figure}[H]
        \centering
        \includegraphics[scale=1]{consultan_pizza.png}
        \label{fig:consultan_pizza}
    \end{figure}

    \item \textbf{\large{Carga Pizza}}

    \begin{figure}[H]
        \centering
        \includegraphics[scale=1]{cargan_pizza.png}
        \label{fig:cargan_pizza}
    \end{figure}

    \item \textbf{\large{Grafo}}

    \begin{figure}[H]
        \centering
        \includegraphics[scale=0.3]{grafo_pizza.png}
        \label{fig:grafo_pizza}
    \end{figure}

\end{itemize}

\subsubsection{\Large{Arco Lleva}}

\begin{itemize}

    \item \textbf{\large{Consulta Lleva}}

    \begin{figure}[H]
        \centering
        \includegraphics[scale=1]{consultan_lleva.png}
        \label{fig:consultan_lleva}
    \end{figure}

    \item \textbf{\large{Carga Lleva}}

    \begin{figure}[H]
        \centering
        \includegraphics[scale=1]{cargan_lleva.png}
        \label{fig:cargan_lleva}
    \end{figure}

\newpage

    \item \textbf{\large{Grafo}}

    \begin{figure}[H]
        \centering
        \includegraphics[scale=0.3]{grafo_lleva.png}
        \label{fig:grafo_lleva}
    \end{figure}

\end{itemize}

\subsubsection{\Large{Nodo Cliente}}

\begin{itemize}

    \item \textbf{\large{Consulta Cliente}}

    \begin{figure}[H]
        \centering
        \includegraphics[scale=1]{consultan_cliente.png}
        \label{fig:consultan_cliente}
    \end{figure}

    \item \textbf{\large{Carga Cliente}}

    \begin{figure}[H]
        \centering
        \includegraphics[scale=1]{cargan_cliente.png}
        \label{fig:cargan_cliente}
    \end{figure}

    \item \textbf{\large{Consulta Teléfonos}}

    \begin{figure}[H]
        \centering
        \includegraphics[scale=1]{consultan_telefonos.png}
        \label{fig:consultan_telefonos}
    \end{figure}

\newpage

    \item \textbf{\large{Carga Teléfonos}}

    \begin{figure}[H]
        \centering
        \includegraphics[scale=1]{cargan_telefonos.png}
        \label{fig:cargan_telefonos}
    \end{figure}

    \item \textbf{\large{Grafo}}

    \begin{figure}[H]
        \centering
        \includegraphics[scale=0.3]{grafo_cliente.png}
        \label{fig:grafo_cliente}
    \end{figure}

\end{itemize}

\subsubsection{\Large{Nodo Pedido y Arco Realiza}}

\begin{itemize}

    \item \textbf{\large{Consulta Pedido / Realiza}}

    \begin{figure}[H]
        \centering
        \includegraphics[scale=1]{consultan_pedido.png}
        \label{fig:consultan_pedido}
    \end{figure}

    \item \textbf{\large{Carga Pedido / Realiza}}

    \begin{figure}[H]
        \centering
        \includegraphics[scale=1]{cargan_pedido.png}
        \label{fig:cargan_pedido}
    \end{figure}

\newpage

    \item \textbf{\large{Grafo}}

    \begin{figure}[H]
        \centering
        \includegraphics[scale=0.3]{grafo_pedido.png}
        \label{fig:grafo_pedido}
    \end{figure}

\end{itemize}

\subsubsection{\Large{Arco Contiene}}

\begin{itemize}

    \item \textbf{\large{Consulta Contiene}}

    \begin{figure}[H]
        \centering
        \includegraphics[scale=1]{consultan_contenido.png}
        \label{fig:consultan_contenido}
    \end{figure}

    \item \textbf{\large{Carga Contiene}}

    \begin{figure}[H]
        \centering
        \includegraphics[scale=1]{cargan_contenido.png}
        \label{fig:cargan_contenido}
    \end{figure}

    \item \textbf{\large{Grafo}}

    \begin{figure}[H]
        \centering
        \includegraphics[scale=0.3]{grafo_contenido.png}
        \label{fig:grafo_contenido}
    \end{figure}

\end{itemize}

\newpage

\subsubsection{\Large{Nodo Membresía}}

\begin{itemize}

    \item \textbf{\large{Consulta Membresía}}

    \begin{figure}[H]
        \centering
        \includegraphics[scale=1]{consultan_membresia.png}
        \label{fig:consultan_membresia}
    \end{figure}

    \item \textbf{\large{Carga Membresía}}

    \begin{figure}[H]
        \centering
        \includegraphics[scale=1]{cargan_membresia.png}
        \label{fig:cargan_membresia}
    \end{figure}

    \item \textbf{\large{Grafo}}

    \begin{figure}[H]
        \centering
        \includegraphics[scale=0.3]{grafo_membresia.png}
        \label{fig:grafo_membresia}
    \end{figure}

\end{itemize}

\subsubsection{\Large{Arco Tiene}}

\begin{itemize}

    \item \textbf{\large{Consulta Tiene}}

    \begin{figure}[H]
        \centering
        \includegraphics[scale=1]{consultan_tiene.png}
        \label{fig:consultan_tiene}
    \end{figure}

\newpage

    \item \textbf{\large{Carga Tiene}}

    \begin{figure}[H]
        \centering
        \includegraphics[scale=1]{cargan_tiene.png}
        \label{fig:cargan_tiene}
    \end{figure}

    \item \textbf{\large{Grafo}}

    \begin{figure}[H]
        \centering
        \includegraphics[scale=0.3]{grafo_tiene.png}
        \label{fig:grafo_tiene}
    \end{figure}

\end{itemize}

\subsubsection{\Large{Nodo Vehículo}}

\begin{itemize}

    \item \textbf{\large{Consulta Vehículo}}

    \begin{figure}[H]
        \centering
        \includegraphics[scale=1]{consultan_vehiculo.png}
        \label{fig:consultan_vehiculo}
    \end{figure}

    \item \textbf{\large{Carga Vehículo}}

    \begin{figure}[H]
        \centering
        \includegraphics[scale=1]{cargan_vehiculo.png}
        \label{fig:cargan_vehiculo}
    \end{figure}

\newpage

    \item \textbf{\large{Grafo}}

    \begin{figure}[H]
        \centering
        \includegraphics[scale=0.3]{grafo_vehiculo.png}
        \label{fig:grafo_vehiculo}
    \end{figure}

\end{itemize}

\subsubsection{\Large{Nodo Conducir y Arcos Conducido\_por y Conduce}}

\begin{itemize}

    \item \textbf{\large{Consulta Conducir / Conducido\_por / Conduce}}

    \begin{figure}[H]
        \centering
        \includegraphics[scale=1]{consultan_conducir.png}
        \label{fig:consultan_conducir}
    \end{figure}

    \item \textbf{\large{Carga Conducir / Conducido\_por / Conduce}}

    \begin{figure}[H]
        \centering
        \includegraphics[scale=1]{cargan_conducir.png}
        \label{fig:cargan_conducir}
    \end{figure}

\newpage

    \item \textbf{\large{Grafo}}

    \begin{figure}[H]
        \centering
        \includegraphics[scale=0.3]{grafo_conducir.png}
        \label{fig:grafo_conducir}
    \end{figure}

\end{itemize}

\subsubsection{\Large{Nodo Transporte y Arco Recorre}}

\begin{itemize}

    \item \textbf{\large{Consulta Transporte / Recorre}}

    \begin{figure}[H]
        \centering
        \includegraphics[scale=1]{consultan_transporte.png}
        \label{fig:consultan_transporte}
    \end{figure}

    \item \textbf{\large{Carga Transporte / Recorre}}

    \begin{figure}[H]
        \centering
        \includegraphics[scale=1]{cargan_transporte.png}
        \label{fig:cargan_transporte}
    \end{figure}

\newpage

    \item \textbf{\large{Grafo}}

    \begin{figure}[H]
        \centering
        \includegraphics[scale=0.3]{grafo_transporte.png}
        \label{fig:grafo_transporte}
    \end{figure}

\end{itemize}

\subsubsection{\Large{Arco Entrega}}

\begin{itemize}

    \item \textbf{\large{Consulta Entrega}}

    \begin{figure}[H]
        \centering
        \includegraphics[scale=1]{consultan_entrega.png}
        \label{fig:consultan_entrega}
    \end{figure}

    \item \textbf{\large{Carga Entrega}}

    \begin{figure}[H]
        \centering
        \includegraphics[scale=1]{cargan_entrega.png}
        \label{fig:cargan_entrega}
    \end{figure}

\newpage

    \item \textbf{\large{Grafo}}

    \begin{figure}[H]
        \centering
        \includegraphics[scale=0.3]{grafo_entrega.png}
        \label{fig:grafo_entrega}
    \end{figure}

\end{itemize}

\newpage

\subsection{\huge{Tercera Etapa. Consultas en Cypher}}

\begin{itemize}

    \item \textbf{\large{Consulta Cypher 1}}

        Mostrar el dni, nombre y apellidos de los 5 repartidores que más kilómetros han recorrido ordenados de más a menos kilómetros recorridos; y mostrar también la cantidad de kilómetros recorrida.
        
    \begin{figure}[H]
        \centering
        \includegraphics[scale=0.4]{consulta_cy_1.jpg}
        \label{fig:consulta_cy_1}
    \end{figure}

    \vspace{1cm}
    
    \item \textbf{\large{Consulta Cypher 2}}

        Mostrar el dni, nombre y apellidos de los dependientes que hablen 2 idiomas o más y hayan recibido al menos 5 suministros.
        
    \begin{figure}[H]
        \centering
        \includegraphics[scale=0.4]{consulta_cy_2.jpg}
        \label{fig:consulta_cy_2}
    \end{figure}

    \newpage
    
    \item \textbf{\large{Consulta Cypher 3}}

        Mostrar el dni, nombre y apellidos de los repartidores activos que lleven menos de 10 años contratados cuyos pedidos entregados tengan al menos un pedido con 9 de valoración; y mostrar también cuantos pedidos tiene cada repartidor con 9 de valoración, ordenados por esta cantidad descendentemente.
        
    \begin{figure}[H]
        \centering
        \includegraphics[scale=0.4]{consulta_cy_3.jpg}
        \label{fig:consulta_cy_3}
    \end{figure}

    \vspace{1cm}
    
    \item \textbf{\large{Consulta Cypher 4}}

        Mostrar el nombre de los ingredientes que están en al menos 6 tipos de pizzas diferentes, mostrando también la cantidad de pizzas en las que están y en orden alfabético de los ingredientes.
        
    \begin{figure}[H]
        \centering
        \includegraphics[scale=0.4]{consulta_cy_4.jpg}
        \label{fig:consulta_cy_4}
    \end{figure}

    \newpage

    \item \textbf{\large{Consulta Cypher 5}}

        Mostrar el dni, nombre y apellidos de cada repartidor ordenados por apellidos, junto con la matrícula y el modelo del vehículo con el que han recorrido más kilómetros, mostrando también los kilómetros recorridos con ese transporte.
        
    \begin{figure}[H]
        \centering
        \includegraphics[scale=0.4]{consulta_cy_5.jpg}
        \label{fig:consulta_cy_5}
    \end{figure}

    \vspace{1cm}

    \item \textbf{\large{Consulta Cypher 6}}

        Mostrar el dni, nombre y apellidos de los clientes que han tenido membresía en más de 5 meses y todas han sido del mismo tipo, mostrando también el tipo de la membresía.
        
    \begin{figure}[H]
        \centering
        \includegraphics[scale=0.4]{consulta_cy_6.jpg}
        \label{fig:consulta_cy_6}
    \end{figure}

    \newpage

    \item \textbf{\large{Consulta Cypher 7}}

        Mostrar el dni, nombre y apellidos de cada cliente que haya pedido una misma pizza al menos 10 veces, junto con el nombre y tamaño de la pizza y las veces que la ha pedido el cliente.
        
    \begin{figure}[H]
        \centering
        \includegraphics[scale=0.4]{consulta_cy_7.jpg}
        \label{fig:consulta_cy_7}
    \end{figure}

    \vspace{1cm}

    \item \textbf{\large{Consulta Cypher 8}}

        Mostrar las 10 pizzas más vendidas a partir de las 10 de la noche hasta las 4 y media de la mañana.
        
    \begin{figure}[H]
        \centering
        \includegraphics[scale=0.4]{consulta_cy_8.jpg}
        \label{fig:consulta_cy_8}
    \end{figure}

\end{itemize}

\end{document}
